
\documentclass{article}


\begin{document}

\title{kikomando}
\maketitle
\section {Executive summary }{This is to enlighten people about how to prepare Kikomando which is one of the common meals  of Uganda`s  population today.Such a meal wouldn`t  have been  attractive, since it was termed as for the less priviledged to many but to my amazement it has turned to be the people`s favorite. 
}
\section {Introduction} {The meal falls in the category of fast foods and  simply it`s  a mixture of sliced chapatti and beans stew.It was started way back before christ,though the preparation style changed. This report gives a clear image on how to prepare Kikomando and make it ready within  few minutes.It also adds more cooking techniques to any in need of more skills. 
}
\section {Background} {First of all,one needs wheat flour according to  the size of meal he wants to prepare, water plus some other ingredients like salt, onions, baking flour,cooking oil and beans can be prepared basing on the test of  the cook.Beans as a source will bring out the real test of Kikomando and so we use some spices like onions,tea masala,royco mukyuzi mix,beaf curry,garlic.For easy and fast preparation,one needs to have two fire sources one  for beans and the other for   chapattis respectively.Charcoal stove can be  used locally and a gas stove or electric oven can be used as the fire source.Water mixed with wheat flour,some ghee, some little salt, smashed onions are mixed togetherand the outcome is called dough which is covered and kept for about 25minutes.This is rolled into small ball pieces and alighned for chapati formation.A chapati roller is used to flatten the balls in the size as wanted by the cook.Flying pan with some cooking oil poured on it is placed on the fire source and wait untill the oil is ready for flying the chapatis.It depends on the experience of the person in charge but some people can fly like two to four chapatis at once so as to save time,cooking oil and also cutoff  more power expences.The beans are boiled on a different fire source for about 30 minutes untill they are ready.Pour some little oil in the source pan and spices are  added basing on ones taste for example some sliced onions,garlic,tomatoes and green pepper.Beans are added, mixed and boiled for a few minutes to get a strong beans stew.The time you inject in cannot be compared to other meal types  because chapatis  take between one and two minutes to get ready.

{
As compared to other meals,Kikomando as nicknamed by Ugandans and has been resorted to by many Ugandans especially the youth population which cannot afford the daily diety of local food. And so the government of Uganda recently declared it as one of the food types we have in Uganda.It has contributed to the economical well being of Ugandans as some people have taken it up as an income generating business.It doesnot require too much capital to venture in it incase one is interested. And for that reason,it has added on the government tax base as more individuals are in position to pay taxes.As more people venture into the business of making and selling Kikomando,there`s a boost in the employment sector .It is one of Ugandas cheapest meals that cost in the range of 1000shs to 2000shs.One meal can satisfy anyone within a short time and may take long to desire for another meal.
}

\section {Lessons} {Let me first make this clear: “I am not in any way discouraging massesfrom longing for other food types but trying to show you another side of life as you  never to stay angry yet there`s away out. It`s  availability  almost every corner of the nation and very few preparation expences  has made it accessible by all. 
}
\section {Conclusion}  {As we fight hunger in Uganda,lets think for the way on how we can help Ugandans to access food in different formats(Kikomando) which is more rich in proteins and carbohydrates.
}   
\end{document}









